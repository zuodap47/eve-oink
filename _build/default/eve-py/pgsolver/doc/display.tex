\section{Viewing Parity Games}
\label{sec:viewing}

\pgsolver can display parity games in two different ways: textually and graphically. The former
can be invoked using the command-line option \texttt{-f}.
\begin{verbatim}
    ~/proj/pgsolver> bin/pgsolver -f 
\end{verbatim}
This results in a simple reprinting of the parity game to be solved at the end of \pgsolver's usual
output. The format is the same as the input format described in the previous section.

In order to display parity games graphically one needs the \texttt{graphviz} package, available for
free from
\begin{quotation}
    \url{http://www.graphviz.org/}
\end{quotation}
\pgsolver can create output in the \texttt{dot}-format which can be displayed using \texttt{dotty} from
the \texttt{graphviz} package for example. The relevant command-line parameter is \texttt{-d} with
a filename which tells \pgsolver to write the \texttt{dot} code into that file after solving the game.
Beware that you need to tell \pgsolver explicitly (how) to solve the game. If you omit this, \pgsolver
will parse the input but not solve the game. However, this can be used to display the game as it is.
\begin{verbatim}
    ~/proj/pgsolver> bin/pgsolver -d graph.dot tests/test1.gm
\end{verbatim}
Then, in order to view it, try
\begin{verbatim}
    ~/proj/pgsolver> dotty graph.dot
\end{verbatim}
\pgsolver can also display a game together with the winning information. This happens when you tell it
to create \texttt{dot}-code for the game \emph{and} tell it to solve it.
\begin{verbatim}
    ~/proj/pgsolver> bin/pgsolver -global recursive -d graph.dot tests/test1.gm
\end{verbatim}
Again, the result can be viewed using
\begin{verbatim}
    ~/proj/pgsolver> dotty graph.dot
\end{verbatim}
for example. However, now nodes and some edges are coloured according to the following specification.
\begin{itemize}
\item The winning region for player $0$, i.e.\ all nodes from which he/she can win the game, is coloured 
      green. The winning region for player $1$ is coloured red.
\item An edge coloured green belongs to the positional strategy for player $0$ that is winning on his/her
      winning region. An edge coloured red belongs to player $1$'s respective winning strategy.
\end{itemize}
An example display of a solved parity game is given on the title page.



%%% Local Variables: 
%%% mode: latex
%%% TeX-master: "main"
%%% End: 

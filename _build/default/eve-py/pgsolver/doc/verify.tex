\section{Verifying Strategies}

The problem of \emph{verifying strategies} is to decide whether a given partition $(W'_0, W'_1)$ of a parity 
game $G$ along with two strategies $\sigma_0: W'_0 \cap V_0 \rightarrow V$ and 
$\sigma_1: W'_1 \cap V_1 \rightarrow V$ matches the partition into winning sets, i.e. whether $W'_0 = W_0$ 
and $W'_1 = W_1$ as well as $\sigma_0$ and $\sigma_1$ ensure the win of the respective player in the 
respective winning set.

The verification process basically traverses three phases. The algorithm verifies in the first phase that 
$\sigma_0$ and $\sigma_1$ are welldefined in the sense that a strategy actually uses valid transitions in 
the game graph. In the second phase the algorithm checks whether the strategies stay in their winning 
regions, i.e. $\sigma_i[W'_i] \subseteq W'_i$ for both player $i$, as well as whether the winning regions 
are closed w.r.t. the respective player.

The third phase finally checks whether the given strategies are winning strategies. In order to solve this 
problem, the algorithm computes the subgames $G_i := (G|_{W'_i})|_{\sigma_i}$ induced by the respective sets 
and strategies in question. Note that both $G_i$ are special games, namely one-player games. Hence, they
can be solved as described above.

Since $\sigma_i$ is a winning strategy on $W'_i$ iff $\sigma_i$ is a winning strategy on $G_i$, it suffices 
to check whether the computed winning set $W^{G_i}_i$ correspond with 
$W'_i$, and if they do not, the counter strategy for player $1 - i$ can be used to extract a cycle in $G$ 
following strategy $\sigma_i$ that is won by $1-i$.




%%% Local Variables:
%%% mode: latex
%%% TeX-master: "main"
%%% End:

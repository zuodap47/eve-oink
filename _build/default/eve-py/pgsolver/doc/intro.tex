\section{The Importance of Parity Games}

Parity games are simple two-player games of perfect information played on directed graphs whose
nodes are labeled with priorities. The name \emph{parity game} is due to the fact that the winner
of a play is determined according to the parities (even or odd) of the priorities occurring in
that play. In fact, it is determined by the maximal priority occurring infinitely often. % This
% can be seen as an abstraction of a setting in which an agent attemps to visit certain configurations
% infinitely often unless some other configurations are also visited infinitely often unless some
% other configurations are als visited infinitely often \ldots while another agent is making his/her
% life difficult by steering into different directions from time to time.

Parity games are an interesting object of study in computer science, and the theory of formal languages
and automata in particular, for (at least) the following reasons.
\begin{itemize}
\item They are closely related to other games of infinite duration like mean payoff games,
      discounted payoff games, stochastic games, etc.\ \cite{Jurdzinski/98,purithesis,Stirling95}.
\item They are at the core of other important problems in computer science, for instance, solving a
      parity game is known to be polynomial-time equivalent to model checking for the modal
      $\mu$-calculus \cite{Emerson93a,Stirling95}. 
\item They arise in complementation or determinisation problems for tree automata
      \cite{lncs2500,focs91*368} and in emptiness and word problems for various kinds of (alternating) 
      automata \cite{focs91*368}.
\item Controller synthesis problems can be reduced to satisfiability problems for branching-time logics
      \cite{AVW03} which in turn require the solving of parity games because
      of determinisations of B\"uchi word automata into parity automata 
      \cite{conf/lics/Piterman06,conf/icalp/KahlerW08}.
\item Solving a parity game is one of the rare problems that belongs to the complexity class
      NP$\cap$co-NP and that is not (yet) known to belong to P \cite{Emerson93a}. The variety of algorithms
      that have been invented for solving parity games is surely due to the fact that many people believe
      the problem to be in P.
\end{itemize}

\section{Aim and Content of \pgsolver}

This variety of algorithms has provided a good understanding of the theory of parity games even though
its computational complexity has possibly not yet been determined precisely. However, this theoretical
knowledge is unmatched by the little amount of investigation into practical aspects of solving parity
games. The aim of this project is to provide a platform for this: it should enable the comparison
between different algorithms not just by the Landau-terms for their worst-case time complexities but
by their actual performance on various classes of parity games.

The current version of this tool contains implementations of the following algorithms found in
the literature:
\begin{itemize}
\item the recursive algorithm due to Zielonka \cite{TCS::Zielonka1998},
\item the local model checking algorithm due to Stevens and Stirling \cite{StevensStirling98},
\item the strategy-improvement algorithm due to Jurdzi{\'n}ski and V\"oge \cite{conf/cav/VogeJ00},
\item the strategy-improvement algorithm due to Schewe \cite{conf/csl/Schewe08},
\item the strategy-improvement algorithm reduction to discounted payoff games due to Puri \cite{purithesis},
\item the randomized strategy-improvement algorithm due to Bj{\"o}rklund and Vorobyov \cite{BjoerklundVorobyov/2007},
\item another randomized strategy-improvement algorithm due to Bj{\"o}rklund, Sandberg and Vorobyov \cite{DBLP:conf/stacs/BjorklundSV03},
\item the small progress measures algorithm due to Jurdzi{\'n}ski \cite{Jurdzinski/00},
\item the small progress measures reduction to SAT due to Lange \cite{lange-gdv05},
\item the dominion decomposition algorithm due to Jurdzi{\'n}ski, Paterson and Zwick \cite{JPZ06},
\item the big-step variant of the latter due to Schewe \cite{Schewe/07/Parity}.
\end{itemize}
In addition, there is a new local strategy improvement algorithm by ourselves. Moreover
there is a direct reduction to SAT based on strategy iteration due to Friedmann.

Finally, there is one heuristic solvers. Such solvers are sound: the answers they provide are correct.
But they are not necessarily complete, for example because they may not terminate. Our heuristic
algorithm just guesses strategies for both players until a (partial) winning strategy has been found.

The heuristic is included because it can solve certain classes of parity games very quickly and
-- most importantly -- in a time that is independent of the number of priorities present in the game.
On the other hand, it is easy to construct games on which it does not terminate, resp.\ infinite families
of games on which the probability of termination decreases exponentially.



\section{Structure of this Report}

Chapter~\ref{chp:pgames} formally introduces parity games and standard notions around the problem of solving them
like winning regions and strategies, but also others that are needed in order to understand the constructions
implemented in various solvers like attractor strategies, decompositions into subgames, etc. It then
describes implemented meta-level optimisations for solving parity games, i.e.\ optimisations that apply to
\emph{any} solver. Next, it shortly describes the implemented algorithms and heuristics. For those known
before we refer to the corresponding literature for a detailed introduction into these algorithms. Here we
only want to point out the rough functionality in order to be able to compare these algorithms and possibly
attribute slow/fast solving to certain techniques.

Chapter~\ref{chp:uguide} is the user's guide. It describes how to compile, install and run \pgsolver, as well
as how to specify parity games that it takes as input and how to read its output, etc.

\pgsolver comes with programs that generate benchmarks. These are for example random games, games
constructed in a way such that they are difficult for a certain algorithm to solve, or application-oriented
games. These are described in Chapter~\ref{chp:benchmarks}.

Finally, Chapter~\ref{chp:devguide} contains the developer's guide. It explains how to
integrate another parity game solver -- implemented in OCaml -- into this tool.

%%% Local Variables:
%%% mode: latex
%%% TeX-master: "main"
%%% End:
